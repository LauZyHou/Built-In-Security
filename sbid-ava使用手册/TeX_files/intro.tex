\chapter{介绍}

\section{功能}
sbid是状态机和序列图相融合的应用软件形式建模工具,并能用于开放网络环境下应用软件安全威胁和安全策略建模。在集成底层的验证工具后可对模型的功能安全性质和信息安全性质进行形式化验证。
\par
在实际使用时,sbid需要和后端验证工具/代码生成工具集成,sbid自身的主要功能如下:
    \begin{itemize}
	\item{软件进程模板的类图建模和状态机建模}
	\item{功能安全性质和信息安全性质规约}
	\item{基于攻击树的威胁建模与威胁判定}
	\item{从抽象语法树生成CTL公式}
	\item{面向对象的序列图建模}
	\item{多结点网络拓扑图建模}
	\end{itemize}
\par
除此之外,在集成相关的后端工具后,sbid才可使用模型检验和代码生成等功能。

\section{关于sbid}
sbid项目(图形界面建模部分)于2019年9月在华东师范大学启动,是使用基于.Net Core 3.0的Avalonia用户界面框架开发的跨平台桌面应用,可以在Windows、OSX和Linux(GTK)上运行。
\par
在Windows下直接运行目录中的sbid.exe文件即可,在OSX或Linux中可从命令行运行sbid命令以启动程序。

