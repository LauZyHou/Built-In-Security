\chapter{项目简介}
\section{研究背景}
\par
随着分布式计算、软件定义网络和自动驾驶等先进技术的快速发展,及其在军事、航天、通信等领域的广泛应用,计算机应用软件已经逐步成为当今信息系统的重要组成部分。无论是社会生产,还是军事防务都越来越多地依赖于应用软件系统来控制。特别地,在人工智能技术的发展趋势下,应用软件为实现万物互联、智能控制等提供了可能,此外,军事信息系统的主要业务功能也将更加依赖计算机应用软件。作为计算服务的直接提供者,应用软件已经成为信息时代的关键基础设施,成为未来网络作战的重要目标对象。但是由于军事领域的特殊性,其对使用的应用软件必须要有极高的安全性保障。一旦应用软件出现安全问题,将会对军事信息系统造成非常严重的后果。因此一方面,确保应用软件的安全性(包括了功能安全和信息安全)对保障军事信息系统的安全性具有至关重要的作用。另一方面,在网络化和智能化技术的发展趋势下,军事信息系统呈现出分布性、并发性、异构性、实时性等多种特性,导致应用软件的架构和代码规模均达到空前的复杂度。与此同时,应用软件的高度复杂性对其安全性保障也提出了巨大的挑战。
\par
设计安全的应用软件是一项复杂的工作,设计者和开发人员的大部分工作在于交付可运行的应用软件,另外安全问题只有在应用软件发现严重的漏洞时才被考虑。避免这些问题的一种方法是采用内生安全应用构造技术,内生安全应用构造技术的含义是指在计算机应用软件的设计开发过程中,将软件构造与安全性机制设计紧密耦合,并通过采用形式化的建模验证与代码生成技术,保证应用软件实现满足相应的安全性需求。利用内生安全应用构造技术,我们将应用软件的安全防御起点提前至软件的设计开发阶段,实现在应用软件研制开发过程阶段有效避免设计缺陷和实现漏洞的出现,从而确保安全性成为应用软件的内生属性,另外这种技术是由安全性要求驱动,并描述了体系结构设计与安全性要求之间的结构化协作和相互关系,以支持系统的长期需求。为此,安全架构的一个关键方面时作为安全工具设计,来提供一个更现实的安全需求软件框架和可以具体识别和执行的需求。
\par
从这些角度来看,应用软件的活动,安全需求工程变得更具挑战与严峻。这些挑战来自架构设计、功能和非功能需求之间的紧密联系,与之相互之间的影响。比如说改变或者更行系统架构设计,安全需求需要满足新的架构设计目标。安全的应用软件需要应对恶意的攻击,因此,在识别需求的阶段就需要考虑这些抽象的安全需求,即不紧紧是细化安全需求和将安全需求联系到安全属性上。因此,在进行应用软件设计之前,首先应该清晰地区别功能和非功能安全问题。
\par
通常来看,军事信息系统是软件和硬件的结合,从而构成一些庞大的系统来完成特定的任务。更加准确地来讲,分布式嵌入式军事信息系统不同于通用的系统在于这些系统严格约束资源在他们的能力范围内,比如说它们的防御。这些系统伴随着可靠性和性能问题,以及实时计算约束。另外安全需求工程方法可分为面向目的(KAOS、Secure Tropos)、面向模型(UMLsec、SecureUML)、面向问题(Abuse Frames、Misuse cases)以及面向过程(SQUARE),KAOS能够引出安全目标来对抗反目标,Secure Tropos来识别恶意对象的目标和计划,分析每个对象的安全约束。UMLsec能够细化安全需求到安全机制与验证,SecureUML是别克认证的约束。SQUARE包括无用的用例,攻击场景,目标以及提出潜在危险的安全需求。结合这些方法,我们提出了将形式化方法与应用软件建模与规约方面相结合,通过采用特定的数学语言对软件功能或行为进行抽象而精确地描述来实施,即对系统的需求进行建模,从而能够支持对安全,复杂,分布式实时的应用软件系统建模。
\section{研究现状}
\par
针对软件缺陷和漏洞问题,常用的方法是通过软件测试发现缺陷,然后对缺陷进行修复。但测试是不完备的,测试只能发现系统中存在的问题,无法证明系统中没有问题。同时,随着软件规模的增大,软件分布性、并发性、异构性和实时性的增强,软件变得越来越复杂,基于测试的方法难以对复杂系统进行完全彻底的检查,也无法发现此类系统中隐藏较深的缺陷与错误。因此,对一些安全攸关的软件系统,尤其是面向军事应用的核心信息系统,必须借助更严格的技术方法,才能保障其安全、正确、可靠的需求。
\par
形式化方法建立在数理逻辑、自动机、形式语义、类型系统等数学理论基础之上,能够以更严格的方式证明系统的安全性、正确性和可靠性。与测试方法相比,形式化方法具有完备性。通过形式化方法,系统的设计要求能够用预先定义好的符号精确地、无歧义地表示成形式规范。形式验证能够以数学推理或状态遍历的方法精确地判定一个系统模型是否满足给定的形式规范,从而揭示软件系统中可能存在的不连贯性、歧义性以及不完全性。
\par
形式化方法作为计算机科学的传统研究方向,在计算机学术界具有重要地位。迄今为止,共有12位学者全部或者部分的因为在形式化方法方面的贡献而获得计算机领域的最高奖——图灵奖。形式化方法在国内外计算机组织中受到普遍重视。例如,国际组织IFIP下面有至少三个工作组(Working Group)与形式化方法相关,欧洲有专门的形式化方法组织FME,美国计算机学会(ACM)在2014年成立了SIGLOG,即逻辑与计算专委会(Special Interest Group on Logic and Computation),涵盖计算逻辑、自动机理论、形式语义、程序验证等方向,而中国计算机学会则在2015年成立了形式化方法专业委员会,由中科院院士林惠民研究员担任专委会主任。由于形式化方法在保证计算机软硬件系统的功能安全和信息安全方面的有效性,已经被许多国际标准化组织列为保证安全攸关系统必备的技术手段。例如,国际航空软件标准DO178B、DO178C中明确要求开发安全可靠航空软件必须使用形式化方法;又如,在软件安全等级SIL1-4中,安全级别最高的SIL3和SIL4要求必须使用形式化方法。
\par
随着网络化和智能化技术的发展,应用软件运行环境逐步呈现开放性、分布性、并发性、异构性等特征,应用软件的复杂度不断增加,形式化方法在应用方面依然面临巨大的技术挑战。
\par
首先,在软件建模与规约方面,传统的基于自然语言和非形式文本的需求描述与分析技术难以从深层次捕捉到应用软件的需求。形式化建模方法能够通过采用特定的数学语言对软件功能或行为进行抽象而精确地描述来实施,即对系统的需求进行建模。当前,尽管存在多种软件建模和规约语言,但是具有严格形式化语义的建模和规约语言对于普通软件工程人员而言难度大,缺乏可读性,直接影响不同背景工程师直接沟通,导致需求模型往往缺失很多重要内容,而UML和AADL等图形化符号又因缺乏严格的形式化语义难以胜任精确化需求描述。
\par
由于认识到形式化方法的重要性,国内外科研管理机构设立了一系列研究计划支持形式化方法的相关研究。
\subsection{国外研究现状}
\par
美国自然基金委于2016年设立“计算探险”重大计划,其中一个项目称为“深度规约的科学(The Science of Deep Specification)”,参与单位包括普林斯顿大学、麻省理工学院、耶鲁大学、宾夕法尼亚大学等许多美国顶尖大学,工业合作方包括微软、谷歌、亚马逊、Facebook等互联网或软件公司。该项目的主要目标是通过形式验证确保软硬件系统的完全功能正确性。其中,耶鲁大学采用形式验证开发了一个全新的操作系统CertiKOS,号称是世界上第一个“没有BUG”的反黑客攻击操作系统。CertiKOS完全使用形式化方法,来开发一个全新的高安全操作系统,并在程序开发的过程中紧密结合形式验证方法提供程序正确性的证明。
\par
2012年,美军国防高级研究计划局(DARPA)启动了“高可信军事网络系统(High-Assurance Cyber Military Systems,HACMS)”项目,目标是研制开发高安全可靠的黑客无法入侵的军事网络系统。该项目的最重要理念是在系统构造过程中引进形式化方法,在系统设计之初,首先严格定义网络系统的需求规范,保证需求规范的正确性与一致性,同时,采用定理证明辅助工具来严格证明网络系统研制过程中系统模型、代码与需求规范的一致性。这与传统被动防御不同,该项目的理念就是在最初设计该网络系统时,就要保证网络系统的功能安全和信息安全,从而消除潜在安全威胁。
\par
2012年,美军国防高级研究计划局(DARPA)启动了“高可信军事网络系统(High-Assurance Cyber Military Systems,HACMS)”项目,目标是研制开发高安全可靠的黑客无法入侵的军事网络系统。该项目的最重要理念是在系统构造过程中引进形式化方法,在系统设计之初,首先严格定义网络系统的需求规范,保证需求规范的正确性与一致性,同时,采用定理证明辅助工具来严格证明网络系统研制过程中系统模型、代码与需求规范的一致性。这与传统被动防御不同,该项目的理念就是在最初设计该网络系统时,就要保证网络系统的功能安全和信息安全,从而消除潜在安全威胁。
\par
作为HACMS的后续项目,美军国防高级研究计划局(DARPA)于2018年又启动了“高可信网络系统工程(Cyber Assured System Engineering,CASE)”,旨在研究不同于传统系统构造方法的高安全可靠网络系统构造技术,摆脱“先构造、再防护”的安全防护技术路线,从软件开发生命周期的角度,通过严密的需求分析、系统软硬件设计、系统集成、测试与验证,保证开发出的系统具备抵抗网络攻击的能力。CASE项目采用设计验证一体化的模式,引导设计者实现高安全可靠网络系统的设计,其主要通过三个途径来实现高安全可靠系统的设计:一是提供可扩展的形式化方法工具来证明系统设计符合需求规范,并生成验证设计模型安全性的测试集;二是开发一个支持高安全可靠性的设计模式库;三是开发生成高安全可靠网络系统监视器的工具。
\par
此外,在工业界,随着形式化方法技术和工具的不断成熟,形式化方法也越来越被著名信息技术公司接受和采纳,并集成到其开发流程中。例如,在国际上,微软开发了一系列形式化验证工具,比如Z3、HAVOC、SLAM等,并将其应用于Windows系统的开发过程之中。微软还基于形式化方法开展了“珠穆朗玛峰”项目,旨在创建一个经过形式化验证的HTTPS软件栈,以有效地保护被称之为“互联网的阿喀琉斯之踵”的网络浏览器。Facebook将基于分离逻辑的验证工具Infer广泛用于在Android应用的开发过程中。Amazon在2014年成立了自动推理组,使用定理证明辅助工具验证其Web服务的正确性。
\subsection{国内研究现状}
\par
国内在基于形式化方法的软件系统构造方面也开展了大量研究工作,但主要侧重在理论方法研究,在工业方面与国外有较大差距。
\par
中国国家自然科学基金委员会于2007年启动了“可信软件基础研究”重大研究计划,累积投入经费1.9亿元,至2016年底结题。该重大研究计划实施的十年间,研究人员以国家关键应用领域中软件可信性问题为主攻目标,既有力推动了软件基础理论的探索与创新,又有效应对了软件发展的重要科学挑战,对促进我国软件产业的振兴与发展具有重大的现实意义。中国科学院软件研究所承担的973课题“安全攸关软件系统的共性理论和质量保障方法”,运用形式化方法,对安全攸关软件的建模、构造、评测、运行、演化等关键问题进行研究。清华大学以嵌入式应用为背景,基于形式化方法的理论基础,开展了可信嵌入式软件系统的建模和验证研究。北京控制工程研究所瞄准航天领域的重大工程,开展了可信嵌入式软件构造与验证的关键技术研究。国内的华为公司组建了形式验证团队,使用定理证明辅助工具来验证其下一代操作系统内核的正确性和安全性。
\par
综上所述,形式化方法是保障软件系统安全可靠性的重要手段,在计算机科学理论界具有重要的核心地位,在国内外科研机构和工业界得到了广泛关注,并已经有了许多成功的应用。在面向军事应用的软件系统开发中,必须借助形式化方法,才能更严格的对其安全可靠性进行保障,实现内生安全的应用软件构造。
\subsection{安全需求工程方法论}
\par
安全需求工程是安全系统中架构设计最重要的部分,为之提供技术,方法和标准来完成系统开发周期中的任务。安全需求工程框架来自借鉴多种安全规约概念的安全需求和安全工程模式,比如安全需求来自试图攻击系统潜在的攻击者。另外,安全需求是来自分析交互和依赖关系的系统模型和攻击的对象。这些不同的观点在不同类型的安全需求和安全设计方案中捕获特定类型的信息和结果。然而,不同的应用软件系统架构层之间的紧密关系需要安全工程师考虑协作行为的方法来提取和执行安全需求。这就要求安全需求工程框架将问题空间的分析扩展到解空间。然而,安全需求不仅仅是识别、优先和细化。安全需求的追踪也是非常重要的问题,提供一个细粒度需求的解决方案对理解给定需求是否必要,如果需求是关联到运行环境、攻击者、甚至是系统架构的改变至关重要。我们需要将从需求识别要求执行,验证和测试结合起来。建立需求和后续工程的紧密联系应该得到这些支持:尽可能的记录下那些采用一些安全机制来满足某个安全需求,和采用一些应用软件实现的测试来验证是否符合同样的需求。也就是说,安全需求应该由这些最抽象的系统行为的文档组成。这些需求应该提供满足系统分析设计实现和验证测试所有阶段的规约。这导致的另一个重要的安全需求工程挑战是应对不一致和不完整的安全需求规范。
\subsubsection{面向目的方法}
\par
面向目的的方法关注于目的的概念,或者引出、阐释、组织、指定和修改安全需求的目的。面向目的的方法两个框架:
\par
KAOS:是第一个以面向目的的方法来建模、指定和分析需求为特征的框架。KAOS是一个关于阐释系统目的的需求工程方法。此外,KAOS将系统的多涉众和多视图考虑在内。KAOS的主要目的是识别高级别的目的和将它逐步细化到精细的操作语义中。默认使用半形式化的图形注解来表示其中的需求。然而,这种方法的主要限制是在最高层次的抽象,系统行为只是以关注主体/客体的特定功能的分析来表示。最新的KAOS需求定义使用了线性时态逻辑。更确切地说,由于行为的输出是一组必需的保护机制和约束系统集合,建立架构设计与安全需求之间的紧密关系,安全需求和安全机制对其他需求的影响是不可缺少的。
\par
Secure Tropos:是Tropos的扩展,将安全问题加入到面向目的过程中。Tropos定义了四个需求开发阶段,在每个连续阶段中将之前阶段的高级别的描述细化到低级别。在Secure Tropos中,安全约束,安全依赖,威胁和安全目的,任务和资源加入到Tropos的建模注解中。此外,安全需求描述为功能上的约束,安全问题集成到Tropos面向代理方法论的所有阶段。在需求的开始阶段,构造涉众的安全图例和安全约束,最后阶段,安全约束应用到系统的安全图例中。系统有一个或者多个用户表示。这种方法遵循一个循序渐进的细化构造,目标是制定在不同级别的抽象上,从高层、策略问题到低级、技术问题。然而这个框架的一个问题是它不提供传播的变化在不同的抽象级别之间。
\par
总的来说,面向目的方法是一种自然的表达方式,细化到其他更抽象的安全需求。这些方法通常假设系统架构的一个相当静态的模型。面向目的方法的另一个优点在于捕捉安全需求之间的依赖关系的能力;
\subsubsection{面向模型的方法}
\par
基于模型的方法主要凸显出架构的定义。通过在不同抽象级别中描述架构概念来表示安全需求。特别是,由于需要识别系统组件中安全问题或组件间的交互才产生这些安全需求。
\par
UMLsec:是UML的扩展,允许在UML图例中表示安全相关的信息。这种方法主要用于压缩知识,使得开发人员可以在形式的一种广泛使用的设计符号,然后提供形式化演算来检查与UMLsec模版关联的约束是否满足给定的规约。更准确的来说,UMLsec的目的是定义一个通用的模版集和标记集,将安全设计知识压缩到可使用UML图例中。用例驱动过程,通过单独对功能和非功能的需求建模,由目的树来设计系统。目的树是用来记录结果或者设计该动作的理由。UMLsec更加聚焦于将安全需求细化到安全机制和验证中。
\par
SecureUML:是另一种基于UML的建模语言,为了基于UML的安全、分布式应用软件模型驱动部署。SecureUML发挥了基于角色访问控制的优势来指定认证约束。与UMLsec和面向目的的安全需求方法相比,SecureUML更关注于软件部署后续阶段。但它并没有考虑安全目的,作用域知识,潜在的攻击和漏洞分析,仅仅关注认证约束和访问控制需求。SecureUML没有考虑提出安全需求,需求的完备性,需求的细化与可追踪信,以及需求间的冲突。此外,这个方法没有提供系统的方法来建立不同安全元素之间的关系,这对设计应用软件的安全解决方案是至关重要的。因此,SecureUML可以看作指定和设计安全应用软件系统的注解,而不是安全需求工程方法。
\par
基于模型的安全需求工程方法的主要限制在于,它主要关涉到系统架构设计的关系间和关注需要满足的低级安全需求与安全价值间的连系。相反,这个特性还描述了这种方法的主要优势是完全符合应用软件的细粒度的架构设计。
\subsubsection{面向问题的方法}
\par
面向问题的方法来定义安全需求关注威胁的定义和如何从它们的标识中提取安全需求。
\par
Abuse Frames:基于Jackson的问题框架方法,用来分析安全问题来决定安全漏洞,以及驱动安全需求。这种方法介绍反需求的注解来描述恶意用户的行为来颠覆现存的需求。明显和精确的描述促进了对威胁的识别和分析,从而推动引出和阐释安全需求。但是,它并没有提供特定的技术或者方法来处理安全需求细化和追踪。
\par
Mususe cases:是传统用例方法的扩展,用来考虑无用的用例和表示系统部署不需要的行为。无用用例由不参与的用户初始。滥用用例的部署允许安全攻击的识别和关联安全需求在应用软件部署中。尽管,滥用用例不是完整的面向问题的方法因为它同时表示问题与解决方案,在早期的软件部署阶段表示安全问题的方法是流行的。但是,它们局限于安全攻击、分析的需求和推动的用例规约。由场景分析的安全需求的完备性不能被保证。此外,这种方法没有考虑合法性验证、需求的冲突或者安全需求和其他非功能需求的交互。
\par
面向问题的方法的主要缺陷是都需要非常详细的系统架构描述和关于已知的在架构组件上呈现漏洞的知识。当它们满足安全认证后,系统的安全属性期望会被评估,就不太适合用来定义一个全新的安全系统。
\subsubsection{面向过程的方法}
\par
面向过程的方法关注从系统设计中来对安全需求进行分析。这些方法包括识别威胁和漏洞,识别和探索安全需求来解决已识别的弱点,危险的分析和安全属性的验证
\par
SQUARE:安全需求工程的全面方法论。旨在将安全需求集合集成到应用软件的部署过程中。SQUARE强调真实应用软件部署项目的应用,并提供可组织的框架来执行安全需求工程活动。SQUARE共由九个步骤组合而成,提供关于提出,分类,排序安全需求的方法。但是,这种定义没有考虑应用软件操作的属性和上下文行为。此外,安全需求是基于文本的描述,这对讲安全需求与其他系统模型的集成和组织安全需求集到描述在不同复杂度的级别来说是非常困难的。
\par
面向过程的方法结合不同阶段的需求工程,如威胁模型,需求的提出,危机分析和需求的优先次序来达到安全需求的更精确和多元化的描述。然后。应用软件部署方法论的硬编码对安全需求有很大的影响。
\par
总的来说,这些安全需求工程方法包括两个主要的阶段:1、识别安全威胁2、设计缓解策略来消除威胁对资产造成伤害的可能性。然而,如前面所述,应用软件系统的困难是需要一个集成的方法来涵盖从架构到实现的应用软件系统设计所有方面。
\section{研究意义}
\par
随着大数据、云计算、人工智能等信息技术的飞速发展,计算机应用软件已逐步成为国防和军事信息系统等安全关键系统的重要组成部分,应用软件的安全性将直接影响全系统的安全性。特别地,在网络化和智能化的发展趋势下,计算机应用软件呈现出分布性、并发性、异构性、实时性等复杂特性,应用软件架构和规模达到了空前的复杂度,使得应用软件安全性保障越来越具有挑战性。
\par
大多数模型驱动工程的工作在于提供合适的方法论和建模环境来设计可靠,复杂,分布式、实时系统。时间限制的分析、调度、资源分配、和并发性通常由这些环境来处理。相比之下,长期以来安全只有在严重缺陷被发现才被重新考量。尤其是最近,安全才成为一个应用软件的主要关注点。然而, 应用软件的大小、异质性和通信特性使之引人注目的发展一个合适的工程环境更显式地定义安全目标和威胁和实施对策。当提交任何应用软件的设计时需求是否保持一致以及满足设计理念,应用软件的复杂性也将验证。
\par
形式化方法与应用软件建模与规约方面相结合的方法,通过采用特定的数学语言对软件功能或行为进行抽象而精确地描述来实施,对系统的需求进行建模。这种方法能够指定系统工程的设计,并验证系统质量的复杂度、提高工具间系统工程信息交换的能力,并帮助消除系统、软件和其他工程学科之间的语义鸿沟。此外,还允许系统工程师遵循系统维度的开发概念内建模,如系统需求分析、系统行为和系统结构。这种建模规约方法能够很好的解决人工编码安全性无法衡量的问题。系统工程师能够设想应用软件可能遭遇的攻击,并在攻击树图中建模这种攻击手段,从而更加容易分析在每个单独的层次系统结构中的安全水平,将安全机制与安全攻击和不匹配的漏洞一一对应,确保应用软件的安全性。它提供分析系统安全的方法从而可以发现简单和复杂的安全攻击和漏洞在抽象系统的不同级别中。最后,这种建模规约方法知识基于攻击树缓解保持安全攻击规范的过程清晰和易懂,尽量不违背一致性的同时实现可维护性。
\section{研究目标}
\par
内生安全应用软件构造技术应用软件形式化建模与规约技术,为应用软件的构造提供模型和性质描述方法是应用软件开发的一个重要技术,它可以有效地对应用软件的进行功能和非功能需求建模。通过对应用软件形式化建模与规约,我们可以降低架构与规模的复杂度,应对应用软件分布性、并发性、异构性、实时性的挑战,在建模阶段找出应用软件的漏洞和威胁,采取措施来保证应用软件的安全性,准确描述应用软件的控制行为和逻辑。
\par
具体研究目标包括:
\par
(1)针对当前应用软件具有分布性、异构性、并发性和实时性等特征,研究基于状态机的应用软件行为建模方法,准确描述应用软件的控制行为和逻辑。
\par
(2)针对分布式环境下应用软件的交互流程,研究基于序列图的应用软件通讯建模方法,描述对象之间发送消息的时间顺序显示多个对象之间的动态协作。
\par
(3)研究开放网络环境下,应用软件运行环境、威胁、攻击的建模方法,以及用于应对威胁和攻击的信息安全策略的建模方法,支持攻击树建模方法,提供一种正式而条理清晰的方法来描述系统所面临的安全威胁和系统可能受到的多种攻击。
\par
(4)研究基于时序逻辑的功能安全(safety)和信息安全(security)属性的规约语言和方法,支持对机密性、完整性、可用性的协同规约。
\par
(5)研制开发应用软件安全构造集成开发套件中形式化建模模块,为应用软件的安全构造奠定基础。
\section{应用领域}
\par
随着计算机网络技术的飞速发展,计算机应用的分布式特性越来越普遍。分布式应用是包含多个计算结点的应用,结点之间通过消息传递进行通信以合作完成某项任务。分布式应用一般可以分为同步和异步系统。同步系统可以看成异步系统的一种特殊情况。此外,如今信息技术快速发展,分布式网络逐步被推广应用到社会发展的各个方面,也渗透到了人们的日常生活中。然而,一些不法分子甚至敌对势力利用网络现有技术存在的缺陷实施网络威胁行为,通过攻击他人机器获取非法利益。最后,分布式应用的功能安全性一般可以归结为系统运行轨迹的性质。时序逻辑通过几个直观的时序算子的组合来方便地来描述系统运行轨迹的性质。时序逻辑一般分为线性时序逻辑和分叉时序逻辑。线性时序逻辑描述分布式系统一个运行轨迹的性质,而分叉时序逻辑描述分布式系统计算树的性质。
\par
内生安全应用软件形式化建模与规约技术方法,采用通信有限状态机及其扩展来对异步分布式应用进行建模。通信有限状态机可以对分布式应用的计算结点的控制逻辑和结点之间的交互过程同时进行建模。此外,基于严格形式化描述的模型检测技术和逻辑推理引擎生成可能的组合攻击模型,适应不断出现的网络威胁模式,实现对应用软件安全威胁的建模技术。为开发人员进行形式规约提供图形化界面支持,对于常用的一些规约模式设计可定制模版,使得开发人员可以快速准确地对一些典型安全性质进行规约和通过对模版进行定制来快速地进行建模。