\chapter{关键技术}
\section{状态机和序列图相融合的应用软件形式化建模技术}
状态机和序列图是工业界广泛使用的两种经典的形式模型,比如状态机是著名的模型检测技术和建模工具,而序列图已经成为著名的UML建模语言的一部分。状态机和序列图各有特点,相辅相成。在分布式的环境下,应用软件一般包括多个计算结点,结点之间通过消息传递来通信。状态机适合对分布式应用的各个结点的控制逻辑进行描述,而序列图可以对计算结点的交互过程进行建模。本项目基于.Net Core 3.0的Avalonia跨平台应用对各种类图、状态机和序列图所模拟的结点控制逻辑和结点交互进过程进行协同建模。
\section{开放网络环境下应用软件安全威胁和安全策略建模技术}
安全威胁建模的研究对应用软件的安全具有极其重要的价值。安全威胁模型能对整个过程进行结构化和系统化的描述,有助于分析和充分利用应用软件已有的威胁行为的研究结果,进一步提供攻击检测和安全预测的效率。本课题拟研究能够对于开放网络环境下应用软件的功能安全和信息安全分别采用SafetyProperty和SecurityProperty两种类图进行建模描述,从而为应用软件威胁模型的构建和安全属性的形式化验证奠定基础。
\section{基于一阶时序逻辑的应用软件安全性质形式规约技术}
形式规约是对应用软件的安全性质(包括功能安全与信息安全性质)的精确描述,形式规约与形式建模一起为形式验证奠定基础。时序逻辑是针对功能安全的形式规约语言,广泛应用于并发分布式系统的形式化验证中。拟研究能够对应用软件安全性质进行协同描述的一阶时序逻辑语言,为应用软件安全性质的形式化验证奠定基础。
