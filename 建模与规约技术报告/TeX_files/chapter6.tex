\chapter{技术分析与总结}
\par
本文从多个角度对内生安全应用软件建模与规约关键方法进行研究。
\par
首先我们从研究应用软件建模与规约的研究入手,对当前广泛使用的安全需求框架和形式化方法进行分析和研究,包括面向目的、面向模型、面向问题以及面向过程的安全需求工程方法论。通过分析四种方法论的优劣,我们提出应用软件形式化建模语言。采用通信有限状态机及其扩展来对异步分布式应用进行建模,并考虑序列图与通信有限状态机相容作为补充。
\par
随后我们分析了内生安全应用软件建模与规约的重要性,包括SPARDL形式化建模语言对经典的状态图进行了扩展,采用时序逻辑对软件的功能安全需求和性质进行详细的刻画。借助这种方法对分布式的应用软件进行建模与规约,可以帮助设计人员充分理解分布式系统的行为,使得设计人员可以通过对模版进行定制快速的建模。另外通过对开放网络下分布式软件的威胁行为的特征进行描述,以及刻画行为属性,有助于设计人员了解复杂网络威胁行为,并构建对应的威胁模型和分析软件的安全属性,帮助系统的模型以及形式验证奠定了基础。本文是通过对应用软件进行内生安全建模与规约的方法来描述分布式应用的安全性质,进而降低应用软件的复杂度。
\par
接着,我们根据应用软件形式化建模与规约语言的关键技术,给出了形式化与应用软件建模规约相结合的方法。研究了应用软件形式化建模语言、开放网络环境下应用软件安全威胁建模、应用软件安全性质与策略形式规约技术。通过这些技术,首先可以对结点的控制逻辑和结点之间的交互过程进行协同建模,另外为应用软件威胁模型的构建和安全属性的形式化验证奠定基础,最后能够对应用软件的功能安全和信息安全性质进行协同描述的一阶时序逻辑语言,从而为对功能安全和信息安全性质进行形式化验证奠定基础。
\par
最后,我们给出了XXX的案例分析。
\par
综上所述,本文通过对应用软件形式化建模与规约技术进行研究,给出了相应的定义。在该方法的指导下,设计人员可以根据软件的功能需求和安全性需求进行形式化建模,得到软件的系统模型、威胁模型和安全规约,降低了应用软件的复杂度,另外对于形式建模与规约提供图形化界面支持,对一些典型的建模与规约场景形成模版,有助于设计人员可以对分布式应用进行快速准确的建模与规约,形成一套用户友好的应用软件形式化建模与规约工具集。
