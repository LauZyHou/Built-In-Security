\chapter{研究内容}
本项目旨在研究内生安全应用软件构造技术,主要研究内容是应用软件形式化建模与规约技术,为应用软件的构造提供模型和性质描述方法。
\section{建模}

\par
建立应用软件的形式建模是开展安全性分析、验证以及正确代码实现的基础。针对当前应用软件具有分布性、异构性、并发性和实时性等特征,研究基于状态机的应用软件行为建模方法,准确描述应用软件的控制行为和逻辑;研究基于序列图的应用软件通讯建模方法,准确描述分布式环境下应用软件的交互流程;研究开放网络环境下,应用软件运行环境、威胁、攻击的建模方法,以及用于应对威胁和攻击的信息安全策略的建模方法;研制开发应用软件安全构造集成开发套件中形式化建模模块,为应用软件的安全构造奠定基础。
\par

\section{规约}

\par
形式化方法的一个重要研究内容是形式规约(Formal Specification,也称形式规范或形式化描述),它是对程序“做什么”(what to do)的数学描述,是用具有精确语义的形式语言书写的程序功能描述,它是设计和编制程序的出发点,也是验证程序是否正确的依据。对形式规约通常要讨论其一 致性(自身无矛盾)和完备性(是否完全、无遗漏地刻画所要描述的对象)等性质。形式规约的方法主要可分为两类:一类是面向模型的方法也称为系统建模,该方 法通过构造系统的计算模型来刻画系统的不同行为特征;另一类是面向性质的方法也称为性质描述,该方法通过定义系统必须满足的一些性质来描述一个系统。不同 的形式规约方法要求不同的形式规约语言,即用于书写形式规约的语言(也称形式化描述语言),如代数语言OBJ、Clear、ASL、ACT One/Two等;进程代数语言CSP、CCS、π演算等;时序逻辑语言PLTL、CTL、XYZ/E、UNITY、TLA等;这些规约语言由于基于不同 的数学理论及规约方法,因而也千差万别,但它们有一个共同的特点,即每种规约语言均由基本成分和构造成分两部分构成。前者用来描述基本(原子)规约,后者 把基本部分组合成大规约。构造成分是形式规约研究和设计的重点,也是衡量规约语言优劣的主要依据。
\par