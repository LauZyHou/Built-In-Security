\chapter{关键技术}
\section{状态机和序列图相融合的应用软件形式化建模技术}
状态机和序列图是工业界广泛使用的两种经典的形式模型,比如状态机是著名的模型检测技术和建模工具,而序列图已经成为著名的UML建模语言的一部分。状态机和序列图各有特点,相辅相成。在分布式的环境下,应用软件一般包括多个计算结点,结点之间通过消息传递来通信。状态机适合对分布式应用的各个结点的控制逻辑进行描述,而序列图可以对计算结点的交互过程进行建模。本项目基于.Net Core 3.0的Avalonia跨平台应用对各种类图、状态机和序列图所模拟的结点控制逻辑和结点交互进过程进行协同建模。
\section{开放网络环境下应用软件安全威胁和安全策略建模技术}
安全威胁建模的研究对应用软件的安全具有极其重要的价值。安全威胁模型能对整个过程进行结构化和系统化的描述,有助于分析和充分利用应用软件已有的威胁行为的研究结果,进一步提供攻击检测和安全预测的效率。本课题拟研究能够对于开放网络环境下应用软件的功能安全和信息安全分别采用SafetyProperty和SecurityProperty两种类图进行建模描述,从而为应用软件威胁模型的构建和安全属性的形式化验证奠定基础。
\section{基于一阶时序逻辑的应用软件安全性质形式规约技术}
形式规约是对应用软件的安全性质(包括功能安全与信息安全性质)的精确描述,形式规约与形式建模一起为形式验证奠定基础。时序逻辑是针对功能安全的形式规约语言,广泛应用于并发分布式系统的形式化验证中。拟研究能够对应用软件安全性质进行协同描述的一阶时序逻辑语言,为应用软件安全性质的形式化验证奠定基础。
\section{面向分布式并发应用软件的限界模型检测技术}
应用软件的分布式特性使得软件模型包含多个模块,极易产生组合空间爆炸问题。为了应对模型检测的状态爆炸问题,快速高效地找出系统中违背安全属性的反例,拟采用限界模型检测的方法,将模型逐步展开,验证给定的安全属性是否成立,将重点突破基于一阶逻辑的模型编码技术、基于可满足性判定的高效模型检测技术等。
\section{面向无穷状态空间应用软件的抽象精化技术}
应用软件的复杂性使得软件模型具有无穷状态空间。为应对无穷状态空间验证问题,需要选取有效的抽象域,将无穷状态空间映射到有穷状态空间。拟研究基于谓词抽象的验证技术,重点突破谓词发现技术、抽象域状态空间遍历和反例自动生成技术,为实现高效的分布式应用软件模型检测提供技术支撑。
\section{面向参数化应用软件的符号模型检测技术}
由于开放网络环境下,应用软件的拓扑结构具有动态变化特性,软件的计算节点个数具有不确定性。为验证动态拓扑结构的应用软件模型,拟研究面向参数化系统的归纳推理验证技术,通过分析分布式应用软件的模块结构、交互过程,将归纳推理与其它验证技术相结合,重点研究归纳不变式的自动生成技术,并结合验证属性加速归纳不变式的生成算法,提升归纳推理应用于应用软件的验证效率。
\section{模型驱动的应用软件代码自动生成技术}
应用软件具有计算行为复杂和并发交互不确定的特点,导致软件代码存在安全缺陷多、缺陷难以重现、纠错难度大等问题。针对这些问题,拟研究模型驱动的应用软件代码自动生成技术,重点研究融合安全策略的应用软件模型的多层次精化技术,保证软件代码与模型的严格一致性,使得软件代码具备安全策略所定义的安全防护能力。
\section{基于程序综合的应用软件代码自动生成技术}
程序综合技术是一项从用户需求出发,“构造即正确”的代码自动生成技术。在形式化软件模型基础上,设计基于程序综合的应用软件代码自动生成方法,以输入/输出测例刻画用户需求,以程序框架为模版,基于程序空间搜索和SMT求解的方式,实现目标程序语言代码的自动生成。支持安全策略自动转化为代码,避免人工实现可能引入的实现错误,保证最后得到的软件代码的功能正确性与安全可靠性。
