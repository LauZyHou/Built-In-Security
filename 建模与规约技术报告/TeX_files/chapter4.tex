\chapter{项目的先进性和创新点}
\par
在该项目中,我们针对基于计算机应用分布式的特性,以及在统一的框架下对有限状态并发分布式系统进行建模、规约和验证(模型检测)中无法描述无穷状态系统的问题,采用以通信有限状态机为基础来构建形式建模语言,以时序逻辑为基础构建形式规约语言。通信有限状态机由Brand和Zafiropulo提出,可以对消息传递异步系统进行直观的建模,其把每个结点描述为一个含有接送和发送动作的有限状态机,而结点之间通过信道进行通信。针对其表达能力存在的缺陷,往通信有限状态机中加入时钟变量来对这些具有实时要求的分布式应用进行建模。此外,通信有限状态机可以对分布式应用的计算结点的控制逻辑和结点之间的交互过程同时进行建模。计算结点之间的交互过程在分布式应用的设计中处于非常中心的位置,而且这能快速的帮助设计人员理解分布式项目的架构。另外本项目采用序列图作为通信有限状态机的补充,将计算结点之间的交互过程显示地进行建模。由于序列图本质上来讲是在对通信有限状态机的进程之间的交互过程的一种抽象,我们检查序列图和通信有限状态机的相容性
\par
另外,时序逻辑是分布式应用的一种广为接受的形式规约语言,分布式应用的功能安全都可归结为诊断系统运行的轨迹。时序逻辑可以通过描述将来行为的Next和Until算子,以及描述过去行为的Previous和Since算子的组合来描述系统运行轨迹,从而描述分布式应用的功能安全性质。线性时序逻辑描述分布式系统一个运行轨迹的性质,而分叉时序逻辑描述了分布式系统计算树的性质。本项目采用线性时序逻辑和其扩展来进行分布式应用的功能性规约。另外安全性质也可以由此规约为一条或多条系统运行轨迹的性质。
\par
因此,由于通信有限自动机和时序逻辑具有直观、表达能力强、支持自动分析与验证的优点,本项目对于形式建模与规约提供图形化界面支持,对一些典型的建模与规约场景形成模版,有助于设计人员可以对分布式应用进行快速准确的建模与规约,形成一套用户友好的应用软件形式化建模与规约工具集。本创新点属于增量创新。