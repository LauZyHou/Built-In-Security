\chapter{ViewModel子模块}
ViewModel子模块中定义了sbid工具中数据和视图的关联类,即是MVVM模式中的模型-视图层。对于Model子模块中的每个数据结构,在ViewModel子模块中都有对应的关联类,其中集成了Model对象,并添加了一些视图层需要使用的方法和属性,对于这些ViewModel不再一一列举,以下仅说明不与Model子模块中数据类相对应的ViewModel类。

\section{StateMachinePanelVM}
StateMachinePanelVM是状态机绘图面板的ViewModel类,将作为StateMachinePanel的DataContxt。
\par
\begin{table}[h]
	\centering
	\begin{tabular}{|c|c|}
		\hline
		\textbf{公共属性}                 & \textbf{描述} \\ \hline
		NetworkViewModel Network      & 绘图网络        \\ \hline
		StateMachineVM StateMachineVM & 对应的状态机      \\ \hline
	\end{tabular}
\end{table}
\par
状态机面板所集成的NetworkViewModel是用于绘图的、定义结点和连线的容器,除了状态机之外,其它绘图面板也需要集成此对象。
\par
\begin{table}[h]
	\centering
	\begin{tabular}{|c|c|}
		\hline
		\textbf{公共方法}                     & \textbf{描述} \\ \hline
		InitialStateVM CreateInitialState & 创建初始状态      \\ \hline
		StateVM CreateState               & 创建普通状态      \\ \hline
		FinalStateVM CreateFinalState     & 创建终止状态      \\ \hline
	\end{tabular}
\end{table}
\par
在sbid工具中,状态机的初始状态总是名为init的状态,终止状态总是名为final的状态,创建这两类状态时,需要到进程模板的状态机引用表中查询是否已经创建过了,如果创建过了,就直接取出其引用,否则才重新创建。
\par
在创建普通状态时,该接口需要传入要创建的状态名,这时要检查该状态是否已经创建过,如果创建过就直接取出引用来使用。sbid工具中的状态机并不反对状态重名,名字相同的状态将表示同一个状态,只是分开来定义了而已。对逻辑上相同的状态结点,需要使用相同的State对象引用。
\par
\begin{table}[h]
	\centering
	\begin{tabular}{|c|c|}
		\hline
		\textbf{公共方法}                      & \textbf{描述} \\ \hline
		TransitionVM ConnectionDragStarted & 用户开始拖拽连线    \\ \hline
		void ConnectionDragging            & 用户正在拖拽连线    \\ \hline
		void ConnectionDragCompleted       & 用户完成拖拽连线    \\ \hline
	\end{tabular}
\end{table}
\par
当用户从锚点开始拖拽连线时,对于已经有转移关系的锚点,就需要将这个转移关系删除,拖拽时监听鼠标位置实时跟踪箭头。拖拽完成后,还要判断是在锚点上完成还是在空白处完成。如果是在空白处完成,则直接放弃正在创建的转移关系。如果是在锚点处完成,则需要先删除锚点上之前维护的转移关系,再创建新的转移关系。

\section{AttackTreePanelVM}
todo 攻击树相关

\section{AttackNode}
todo 攻击树相关

\section{RelationNode}
todo 攻击树相关


