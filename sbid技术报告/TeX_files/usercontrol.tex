\chapter{UserControl子模块}
UserControl子模块中定义了视图层中的用户自定义控件,主要包含各个面板的视图定义、数据绑定和用户操作事件处理。

\section{StateMachinePanel}
StateMachinePanel是状态机的绘图面板,使用一个套在ScrollViewer中的NetworkView,以实现带滚动条的图形移动和连线的面板。在这里用数据模板的形式为状态机的状态结点、转移关系和结点上的锚点设置了外观样式。
\subsection{转移关系的样式}
TransitionVM的样式需求是箭头的中间有Guard条件和Actions列表。箭头用Arrow图形来实现,起始点和终点绑定两端锚点的位置。而为求得箭头中间位置,对ConnectionViewModel进行了扩充,添加了一个MidConnectorPoint属性,每次由两端锚点的位置计算中心点,当两端锚点位置变化时将通知此属性重新计算位置。然后将Guard条件和Actions列表组合在StackPanel中,再和Arrow同放在以左上角点开始的自适应Canvas中,最终为整个StackPanel设置Canvas.Left和Canvas.Top两个附加属性绑定在计算出的中心点位置即可。其中Guard所在的TextBox的Text和Actions所在的ListBox的ItemsSource要分别绑定TransitionVM所集成的Transition数据对象的Guard和Actions属性。
\subsection{初始状态的样式}
InitialStateVM的样式需求是仅最下方有一个锚点的黑色正圆形。其中黑色正圆形用放在正方形Grid中的Ellipse实现,然后在同一层放置一个9*9的Grid,并在最下方位置绑定锚点。
\subsection{终止状态的样式}
FinalStateVM的样式需求是仅最上方有一个锚点的黑色正圆形。与初始状态的实现一致,不同之处仅是将锚点置于9*9的Grid的最上方。
\subsection{普通状态的样式}
StateVM的样式需求是矩形框中间书写状态名,周围放置锚点。其中矩形框用放置在矩形Grid中的Rectangle实现,然后在同一层放置一个9*9的Grid,并在中间放置一个绑定了状态名称的TextBox,在周围放置锚点。

\section{AttackTreePanel}
todo 攻击树相关